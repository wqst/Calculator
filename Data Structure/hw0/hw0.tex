\documentclass[12pt]{article}
\usepackage[margin=1in]{geometry}
\usepackage[bitstream-charter]{mathdesign}
\usepackage{fancyhdr}

\pagestyle{fancyplain}
\begin{document}
\lhead{hw0}
\rhead{CS 225 Summer 2014}

\thispagestyle{empty}

\begin{center}
\LARGE
\textbf{CS 225 Summer 2014 Homework 0}\\
\Large Due June 18, 2014, 3:00pm in lecture (and on SVN)
\end{center}

\bigskip
\hrule
\begin{quote}
The purpose of this assignment is to give you a chance to refresh the math
skills we expect you to have learned in prior classes.  These particular skills
will be essential to mastery of CS 225, and we are unlikely to take much class
time reminding you how to solve similar problems.

Though you are not required to work independently on this assignment, we
encourage you to do so because we think it may help you diagnose and remedy some
things you might otherwise find difficult later on in the course.
\end{quote}
\hrule
\bigskip

{\Large
    \vspace{1in}

    \textbf{Display your name, netid, and section time (11am, 1pm, or 5pm)
    on the front page of your submission.}

    \vspace{1in}

    \textbf{Your hw0 MUST be typewritten and submitted as a PDF file
    (hw0.pdf)!} You will also hand in a physical, printed copy in lecture.
}

\newpage

\textbf{Submission Instructions}\\

To checkout your code, run the command

\begin{verbatim}
svn co https://subversion.ews.illinois.edu/svn/su14-cs225/NETID/hw0/
\end{verbatim}

This will download two hw0 files: hw0-handout.pdf (this file) and hw0.tex (an
optional \LaTeX template). Create your answers electronically (via Word,
\LaTeX, etc), and produce a PDF file as output. \textbf{\emph{Make sure your
file is called hw0.pdf and the writeup is called writeup.txt}}. When you are
ready to submit, run the following commands:

\begin{verbatim}
svn add hw0.pdf
svn add writeup.txt
svn ci -m "submitting hw0"
\end{verbatim}

You should verify your submission by going to the checkout link in your browser
and making sure the file can be downloaded. You will also bring a printed,
physical copy to lecture on the due date. If you have any further
questions, feel free to ask on Piazza. The questions begin on the next
page.

\newpage

\begin{enumerate}

    \item (4 points) Write a few paragraphs (minimum 300 words) in a file called writeup.txt and commit
    it to your hw0 directory as mentioned in the Instructions. Answer the
    following prompts:

\begin{enumerate}
    \item Tell me what interests you about computer science, and why you are
        fascinated by that particular thing.
    \item Tell me what you hope to do with your particular skills in computer
        science (\emph{e.g.} graphics, databases, natural language processing,
        security)
\end{enumerate}

\item (4 points) Post a short synopsis of your favorite
    movie to the course piazza space under the ``HW0 tell me something!''
    notice, so that your post is visible to everyone in the class, and tagged by
    \#movie. Also, mention someplace interesting you have traveled in a
    private post to course staff, also with the tag \#travel. Finally, please
    record the two post numbers corresponding to your posts.

\begin{enumerate}
    \item Favorite Movie Post (Public) number:
    \item Summer Travel Post (Private)  number:
\end{enumerate}

\item (18 points) Simplify the following expressions as much as possible, Do not
    approximate. Express all rational numbers as improper fractions.
    \textbf{Show your work.}

\begin{enumerate}

\item $\prod_{k=2}^n (1-\frac{1}{k^2})$

\item $3^{1000} \bmod 7$

\item $\sum_{r=1}^\infty(\frac{1}{2})^r$

\item $\frac{\log_7 81}{\log_7 9}$

\item $\log_2 4^{2n}$

\item $\log_{17} 221 - \log_{17} 13$

\end{enumerate}

\item (15 points) Find the formula for $1+$ $\displaystyle\sum_{j=1}^n j!j $, and
    show work proving the formula is correct using induction.

\item (12 points) Indicate for each of the following pairs of expressions
    $(f(n),g(n))$, whether $f(n)$ is  $O, \Omega,$ or $\Theta$ of $g(n)$.  Prove
    your answers to the first two items, but just GIVE an answer to the last
    two.

\begin{enumerate}

\item $f(n)=4^{\log_{4} n}$ and $g(n)=2n+1$

\item $f(n)= n^2$ and $g(n)=(\sqrt{2})^{\log_2 n}$

\item $f(n) =\log_2 n!$ and $g(n) =n \log_2 n$

\item $f(n) = n^k$ and $g( n)=c^n$ where $k,c$ are constants and $c$ is $>$1

\end{enumerate}

\newpage

\item (15 points) Solve the following recurrence relations for integer $n$. If no
    solution exists, please explain the result.

\begin{enumerate}

\item $T(n)=T(\frac{n}{2}) + 5$, $T(1) = 1$, assume $n$ is a power of 2.

\item $T(n)= T(n-1) + \frac{1}{n}$, $T(0) = 0$.

\item Prove that your answer to part (a) is correct using induction.

\end{enumerate}

\item (16 points) Suppose function call parameter passing costs constant time,
    independent of the size of the structure being passed.

\begin{enumerate}

\item Give a recurrence for worst case running time of the recursive Binary
    Search function in terms of $n$, the size of the search array. Assume $n$ is
    a power of 2. Solve the recurrence, explicitly noting the recurrence
    formlua, base case, and recurrence solution.

\item Give a recurrence for worst case running time of the recursive Merge Sort
    function in terms of $n$, the size of the array being sorted. Solve the
    recurrence, explicitly noting the recurrence formlua, base case, and
    recurrence solution.

\end{enumerate}

\item (16 points) Consider the pseudocode function below.

\begin{verbatim}
darp( x, n )
     if( n == 0 )
         return 1;
     if( n % 2 == 0 )
         return darp( x * x, n/2 );
     return x * darp( x * x, (n - 1) / 2);
\end{verbatim}

\begin{enumerate}

\item What is the output when passed the following parameters: $x=2$, $n=12$.
    Show your work (activation diagram or similar).

\item Briefly describe what this function is doing.

\item Write a recurrence that models the running time of this function.  Assume
    checks, returns, and arithmetic are constant time, but be sure to evaluate
    all function calls. [Hint: what is the \emph{most} $n$ could be at each
    level of the recurrence?]

\item Solve the above recurrence for the running time of this function.

\end{enumerate}

\end{enumerate}

\end{document}
